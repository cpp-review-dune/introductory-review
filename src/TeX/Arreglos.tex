\chapter{Arreglos}

Esta teoría es parte del legado de \texttt{C++}.
Un arreglo es un tipo de data compuesto o data estructurada.
Todos los elementos deben ser del mismo tipo y cada elemento puede ser accesado directamente.
Algunas características:

\begin{compactitem}
	\item No pueden cambiar de tamaño una vez creados.
	\item Todos los elementos son del mismo tipo.
	\item Sus elementos son guardados contiguamente en memoria.
	\item Elementos individuales son accesibles por su posición o índice.
	\item Su primer elemento tiene índice cero.
	\item Su último elemento tiene índice \texttt{size-1}.
	\item Es responsabilidad del programador accesar los elementos de un arreglo dentro de sus límites.
	\item Es una buena costumbre inicializar los arreglos.
\end{compactitem}

Al declarar arreglos debemos usar la siguiente sintaxis:
\begin{center}
	\micod{TipoDeElemento NombreDeArreglo [número constante de elementos];}
\end{center}

Veamos algunos ejemplos con inicialización de elementos del arreglo:
\includeformula{1.cc}

Para accesar a los elementos de un arreglo la sintaxis es:
\begin{center}
	\micod{nombre_arreglo[índice_de_elemento]}
\end{center}
Veamos un ejemplo
\includeformula{2.cc}
Usamos la misma sintaxis si deseamos guardar información en el arreglo:
\includeformula{3.cc}
Veamos ahora un programa que resuma lo que hemos visto hasta el momento:
\includecppinput{4.cc}
Luego de compilar este archivo y e ingresar una data cuando el archivo es ejecutado, obtenemos:
\immediate\write18{./programas/run.sh}
\includecppoutput{salida4}

Manejo de Ciclos
\includecppinput{5.cc}
% Para que latex compile el programa 5.cc y lo guarde en el archivo 5
%Luego envía el resultado o la salida al archivo de texto salida5
%Incluye la salida en el pdf
\includecppoutput{salida5}
%Incluir el archivo Arreglos.tex para agregar datos y compilar
%\includetexinput{Arreglos.tex}
\includecppinput{Prueba1cin.cc}
\includecppoutput{salidaPrueba1cin}